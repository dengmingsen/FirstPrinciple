\documentclass[notitlepage]{revtex4-1}
\usepackage{geometry}
\usepackage{graphicx}
\usepackage{times}
\usepackage{physics}   % for simple physics notation
\usepackage{bm}        % for math
\usepackage{amssymb}   % for math
\usepackage{amsmath}
\usepackage{subfigure}
\usepackage{color}
\usepackage{float}
\usepackage{enumerate}
\usepackage{enumitem}
\usepackage[export]{adjustbox}
\usepackage{comment}
\usepackage{listings}
\usepackage{CJK}
\usepackage{graphicx}
\newcommand{\hilight}[1]{\colorbox{red}{#1}}
%\usepackage{physics}
%\usepackage{enumerate}
%\usepackage{booktabs} % not allowed in Revtex4.1
\begin{document}
\begin{CJK}{UTF8}{bsmi}
\title{First Principle 2017-Fall  Homework 2 Solution}
%\input author_list.tex       % D0 authors (remove the first 3 lines
                             % of this file prior to submission, they
                             % contain a time stamp for the authorlist)
                             % (includes institutions and visitors)
\author{Kai-Hsin Wu (吳愷訢)}
\email{r05222003@ntu.edu.tw}
\affiliation{Department of Physics and Center of Theoretical Sciences, National Taiwan University, Taipei 10607, Taiwan}

%\date{\today}
\maketitle

\begin{enumerate}	
	\item We start with Hartree-Fock hamiltonian for N electrons:
	\begin{align*}
		\hat{H}_{HF} &= \sum_{n=1}^{N} \hat{h}_n + \sum_{n=1}^{N}\sum_{m=1}^{N} \frac{1}{|r_n - r_m|} 
	\end{align*} 
	
	Consider the many-electrons wave function $\Psi(x_1,x_2...)$ which can be represented as slater determinant:	
	\begin{equation*}
		\Psi(x_1, x_2, ...) = 
		\begin{vmatrix}
			\phi_{1}(x_1) & \phi_{2}(x_1) & \phi_{3}(x_1) & ... \\
			\phi_{1}(x_2) & \phi_{2}(x_2) & \phi_{3}(x_2) &... \\
			\phi_{1}(x_3) & \phi_{2}(x_3) & \phi_{3}(x_3) &... \\
			... &  ... & ... & ... \\
		\end{vmatrix}
	\end{equation*}

	To derive the Koopermans' theorem, we first remove one electron from orbital with wave vector $k$:
	
	\begin{equation*}
		\hat{H}_{N-1}^{k} = \sum_{n=1,n\neq k}^{N-1} \hat{h}_{n} + \sum_{n=1}^{N-1} \sum_{m=1}^{N-1} \frac{1}{|r_n - r_m|} 
	\end{equation*} 
	
	where 


	
\end{enumerate}




\bibliographystyle{apsrev4-1}
\bibliography{ref}
	
\end{CJK}
\end{document}

