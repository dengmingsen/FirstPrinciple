\documentclass[notitlepage]{revtex4-1}
\usepackage{geometry}
\usepackage{graphicx}
\usepackage{times}
\usepackage{physics}   % for simple physics notation
\usepackage{bm}        % for math
\usepackage{amssymb}   % for math
\usepackage{amsmath}
\usepackage{subfigure}
\usepackage{color}
\usepackage{float}
\usepackage{enumitem}
\usepackage[export]{adjustbox}
\usepackage{comment}
\usepackage{listings}
\usepackage{CJK}
\usepackage{graphicx}
\usepackage{booktabs}
\usepackage{hyperref}
\newcommand{\hilight}[1]{\colorbox{red}{#1}}
%\usepackage{physics}
%\usepackage{enumerate}
%\usepackage{booktabs} % not allowed in Revtex4.1
\begin{document}
\begin{CJK}{UTF8}{bsmi}
\title{First Principle 2017-Fall  midterm Solution}
%\input author_list.tex       % D0 authors (remove the first 3 lines
                             % of this file prior to submission, they
                             % contain a time stamp for the authorlist)
                             % (includes institutions and visitors)
\author{Kai-Hsin Wu (吳愷訢)}
\email{r05222003@ntu.edu.tw}
\affiliation{Department of Physics and Center for Theoretical Sciences, National Taiwan University, Taipei 10607, Taiwan}

%\date{\today}
\maketitle

\begin{enumerate}	
	\item Al band structure using GGA calculation and free-electron band structure.
	
	\item Kohn-Sham
	
	\item Car-Parrinelo EOM
	
	\item GGA, GGA+U of MnO in AF-II

	\item Finite difference algorithms. 
	
		In the following, we evaluate the harmonic oscillator with "Euler" , "Predictor-Corrector" and "Velocity-verlet" method.
		\begin{enumerate}[label=(\alph*)]
		\item Euler method with $dt = 2.5e-3 \pi$, $x(0) = 1$,$v(0) = 0$ 

		\item Euler method with $dt = 2.5e-4 \pi$, $x(0) = 1$,$v(0) = 0$ 
		
		compare the result with (a), we can see that  reducing the update time interval, the energy still not conserved, but the error is decreased. 


		\item Euler method compare with Predictor-Corrector method with $dt = 2.5e-3 \pi$, $x(0) = 1$,$v(0) = 0$ 

		compare the result with (a), we can see that with the same update time interval ($dt$), we can see that the energy increment error is significantly reduced.  

		\item Euler method compare with Velocity-Verlet method with $dt = 2.5e-3 \pi$, $x(0) = 1$,$v(0) = 0$ 
		
		compare the result with (a), we can see that with the same update time interval ($dt$), we can see that the energy increment error is reduced.  

		\end{enumerate}

\end{enumerate}


\end{CJK}

\bibliographystyle{apsrev4-1}
\bibliography{ref}
	

\end{document}

